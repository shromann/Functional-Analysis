\documentclass{article}
\usepackage[a4paper, total={6in, 9in}]{geometry}
\usepackage{graphicx} % Required for inserting images
\usepackage{amsfonts, amsthm, amssymb}
\usepackage[hidelinks]{hyperref}
\usepackage[T1]{fontenc}
\usepackage{palatino}


\newtheorem{definition}{Definition}
\newtheorem{theorem}{Theorem}[section]
\newtheorem{lemma}[definition]{Theorem}
\newtheorem{corollary}[definition]{Corollary}

\renewcommand{\l}[1]{\ell^{#1}}
\renewcommand{\sb}[1]{\left[#1\right]}
\newcommand{\rb}[1]{\left(#1\right)}
\newcommand{\cb}[1]{\left{#1\right}}

\newcommand{\ra}{\rightarrow}
\newcommand{\defn}[2]{\begin{definition}\textbf{#1} #2 \end{definition}}
\newcommand{\thrm}[2]{\begin{theorem}\textbf{#1} #2 \end{theorem}}
\newcommand{\lmma}[2]{\begin{lemma}\textbf{#1} #2 \end{lemma}}
\newcommand{\clry}[2]{\begin{corollary}\textbf{#1} #2 \end{corollary}}

\newcommand{\eS}[2]{#1 \in #2}
\newcommand{\R}{\mathbb{R}}
\newcommand{\inR}[1]{\eS{#1}{\R}}
\newcommand{\C}{\mathbb{C}}
\newcommand{\inC}[1]{\eS{#1}{\C}}
\newcommand{\F}{\mathbb{F}}
\newcommand{\inF}[1]{\eS{#1}{\F}}
\newcommand{\N}{\mathbb{N}}
\newcommand{\inN}[1]{\eS{#1}{\N}}


\newcommand{\V}{$V$}
\newcommand{\inV}[1]{\eS{#1}{V}}

\newcommand{\abs}[1]{\left|#1\right|}
\newcommand{\norm}[1]{\abs{\abs{#1}}}
\newcommand{\map}[3]{#1 : #2 \ra #3}
\newcommand{\set}[2]{\{#1|#2\}}
\newcommand{\seq}[1]{\rb{#1_n}}
\newcommand{\conv}[2]{\seq{#1} \ra #2}
\newcommand{\isFinite}[1]{#1 < \infty}
\newcommand{\sumInf}[1]{\sum_{#1}^\infty}
\newcommand{\sumFin}[2]{\sum_{#1}^{#2}}

\title{Functional Analysis}
\author{Shromann Majumder}
\date{\today}

\begin{document}
\maketitle

Throughout these notes, we will denote \V as a \hyperlink{https://en.wikipedia.org/wiki/Vector_space}{vector space} over a field $\F$, for which is either $\C$ or $\R$ (unless otherwise stated). We present definitions and theorems, along their respective proofs. These notes assume no prior background in pure mathematics, hence all recalled definitions and theorems mentioned in pre-requisite courses are also provided. Welcome to \emph{Functional Analysis}. 

\tableofcontents

\section{Normed and Banach Spaces}
\defn{Norm}{is a map $\map{\norm{\cdot}}{V}{\R^+}$ which satisfies
\begin{enumerate}
    \item Definiteness, $\norm{v} \geq 0$ for $\inV{v}$ (equality holds for $v=0$),
    \item Scalar Multiplication, $\norm{\lambda v} = \abs{\lambda}\norm{v}$ for $\inF{\lambda}$,
    \item Triangle Inequality, $\norm{u + v} \leq \norm{u} + \norm{v}$ for $\inV{u, v}$.
\end{enumerate}
}
\section{Bounded linear operators}
\section{The dual space and the Hahn-Banach theorem}
\section{The spectrum of an operator and functional calculus}
\section{Hilbert spaces}
\section{Linear functionals and linear operators on Hilbert spaces}
\section{The spectrum of an operator and functional calculus}
\section{Compact operators}



\end{document}
